% !TEX program = xelatex
\DocumentMetadata{lang=en} % required for transparent package
\documentclass[11pt,aspectratio=169]{beamer}

% remove footcite numbers
\makeatletter
\def\@makefnmark{}
\makeatletter

\setbeamersize{text margin left=5mm,text margin right=5mm} 

\newcommand\focus[1]{%
	{\alert{\textbf{#1}}}
}

\usepackage{amsthm,amsmath,amssymb,braket,fontspec,unicode-math,fontenc,transparent}
\usepackage[absolute,overlay]{textpos}

% \usetheme{focus}
\usetheme[numbering=none]{focus}
% \setmainfont{Barlow}
% \setsansfont{Barlow}
% \setmathfont{Fira Math}
\setmathfont{latinmodern-math.otf}[range={frak,\bigcap,\bigcup}]

\usepackage[backend=bibtex,url=false,doi=false,style=authoryear]{biblatex}
\setbeamertemplate{bibliography item}{}
\bibliography{bib}
\AtBeginBibliography{\scriptsize}

\graphicspath{{./figures/}}

\setbeamerfont{title}{size=\LARGE\scshape}
\setbeamerfont{author}{size=\Large}
\setbeamerfont{institute}{size=\large}
\setbeamerfont{date}{size=\large}
\setbeamerfont{frametitle}{size=\Large\scshape}
\setbeamerfont{sectiontitle}{size=\small\scshape}

\title{Insights on Mott Transition and The Pseudogap Through the Veil of a Quantum Impurity Model}
\author{Abhirup Mukherjee}
\institute{DPS Day '25 \\ Department of Physical Sciences, IISER Kolkata}
\titlegraphic{
	\vspace{20pt}
	\hfill\includegraphics[width=0.1\textwidth]{epqm_logo_mod.jpeg}\hspace*{20pt}\includegraphics[width=0.1\textwidth]{dps_logo.jpeg}\hspace*{30pt}
}

\begin{document}

\centering

\begin{frame}
\maketitle
\end{frame}

\begin{frame}{Acknowledgements}

	\hspace*{\fill}
	\includegraphics[width=0.1\textwidth]{SERB.png}
	\hspace*{\fill}
	\includegraphics[width=0.1\textwidth]{dps_logo.jpeg}
	\hspace*{\fill}
\end{frame}

\begin{frame}{Quantum Materials Are Different}
\end{frame}

\begin{frame}{Why Do We Need New Methods?}
\end{frame}

\begin{frame}{A New Auxiliary Model Approach}
\end{frame}

\begin{frame}{The Correct Auxiliary Model - A Lattice Embedded Impurity}
\end{frame}

\begin{frame}{Momentum-space Resolved Kondo Breakdown}
\end{frame}

\begin{frame}{Pseudogapping Transition on the Lattice Model}
\end{frame}

\begin{frame}{Nature of The Metal Close to the Transition}
\end{frame}

\begin{frame}{What are the main takeaways?}
\end{frame}

\begin{frame}{Future prospects}
\end{frame}

\end{document}
